\documentclass{article}
\usepackage{amsmath}
\usepackage{xeCJK}
\usepackage{amssymb}
\usepackage{indentfirst}
\setCJKmainfont{標楷體}

\title{Complex Analysis}
\author{107021211鄭竣元}
\date{June 2022}

\begin{document}

\maketitle

\section{Cauchy-Riemann equation}
A function $f(z)=u(x,y)+iv(x,y)$ , $z=x+iy$ is complex differentiable at $z_0=x_0+iy_0$ if and only if $u$ and $v$ are real differentiable at $(x_0,y_0)$.\\
也就是說$u_{x}=v_{y}$且$u_{y}=-v_{x}$ 等價於 $f(z)$ 在 $z_0$ 可微分。
\section{Entire}

$f(x)$ is called entire if it is differentiable in all $\mathbb{C}$ fields.\\
\\
differentiable 是對點($z_0$),analytic 是對區域 (open set),entire 是對整個 $\mathbb{C}$\\
\\
\\
Ex1. show $e^{z}$ is entire.
\\
\begin{align*} 
e^{z} &=  e^{x+iy} \\ 
&=  e^{x}(\cos{y}+i\sin{y})\\
&= e^{x}\cos{y} + ie^{x}\sin{y}\\
\end{align*}
check by Cauchy-Riemann equation $u_{x}=v_{y}$ and $u_{y}=-v_{x}$ is hold on all $\mathbb{R}$ fields.
thus $e^{z}$ is entire.\\
\\
Ex2. show $e^{z}$ is only entire function satisfied
\begin{align*} 
f(z_{1})+f(z_{2}) &=  f(z_{1})f(z_{2}) \\ 
f(x)&=  e^{x}, x\in \mathbb{R} \\
\end{align*}
let $f(z)=g(x,y)+ih(x,y)$\\

\begin{flushleft}
$\because f(z)$ is entire, $f'(0) =e^{0} = 1$\\
$\therefore  g'(0)+ih'(0) = 1$\\
then $g'(0) = 1, h'(0) = 0$\\
\end{flushleft}

thus $g(x,y)=x, h(x,y)=0$\\

反向用Cauchy-Riemann equation,得到:\\

\begin{align*}
g_{x} &= h_{y} \\
g_{y} &= -h_{x} \\
\end{align*}
we get
\begin{align*}
    f''(x) + f(x) &= 0 \\
\end{align*}
所以$f(x) = c_1sin(x) + c_2cos(x)$,(微分兩次等於自己相反也只有sin,cos的產物)\\
帶入$f'(0) = 1$,得到$c_1 = 1, c_2 = i\longrightarrow f(z) = e^{z}$\\

\section{Convergence of complex series}
Thm1\\
$\sum a_{n} z^{n}=A(z)$ is Convergence at radius $R_{1}$\\
$\sum b_{n} z^{n}=B(z)$ is Convergence at radius $R_{2}$\\
$\Rightarrow \sum_{n=0}^{\infty} (\sum_{k=0}^{n} a_{n}\cdot b_{n-k}) z^{n} =A(z) \cdot B(z)$ is Convergence at radius $min(R_{1},R_{2})$\\
proof: 爆破 斜線排在一起 \\

Thm2\\
power series 微分後收斂半徑不變,且無限可微\\
hint: 展開 微分 用原本的$R$作 Convergence test\\

Thm3 (uniqueness)\\
當$f(z)=\sum_{n=0}^{\infty} c_{n} z^{n}$滿足
\begin{enumerate}
    \item 存在非0數列$\{z_{n}\}$,且 $lim_{n \to \infty} z_{n} = 0$\\
    \item 對於所有$z_{n}$, $f(z_{n}) = 0$\\
\end{enumerate}
則$f(z) \equiv 0$ (identical zero)  ($\{c_{n}\}=0$對所有$n$)\\
*推廣:到任何常數 $c$ 也可以 ($g(z)=f(z)+c$)\\
***他告訴了我們 如果兩個analytic function 在一個accumulated point 相等,則兩個function相等\\

Thm3.1: uniqueness of analytic function\\
如果兩個analytic function在一個收斂數列處處相等,則兩個function相等。\\


\section{Accumulated point}
A point $z_0$ is called an accumulated point of a set $E$ if every neighborhood of $z_0$ contains at least one point of $E$ different from $z_0$ itself.\\
(可透過收縮open set 找到一串收斂數列 即$\{d_n\}$ ,$d_{n+1}$ 在半徑為$d(d_{n},z_0)$的openball裡面找)

\section{M-L formula}
if $f(z)$ is continuous on a curve $\gamma$ with length $L$, and if $|f(z)| \leq M$ on $\gamma$, then
\begin{align*}
    |\int_{\gamma} f(z) dz| \leq ML
\end{align*}

\section{analytic function}
幾個analytic function的性質
\begin{enumerate}
    \item $f(z)$ is analytic at $z_0$ if and only if $f(z)$ can be represented by a convergent power series in some neighborhood of $z_0$.
    \item 如果$f(z)$在一個區域$z_0$內解析,則$f(z)$在$z_0$內無限可微。(power series微分後收斂半徑不變,且無限可微)
\end{enumerate}
所以解析函數 == 無限可微函數 = power series (複數空間的微分超強特性)

\section{Cauchy integral formula}
if $f(z)$ is analytic inside and on a simple closed circle $C=R e^{i\theta}$ 半徑$R$的封閉圓圈, then
\begin{align*}
    f(a) = \frac{1}{2\pi i} \int_{\gamma} \frac{f(z)}{z-a} dz \\    
\end{align*}
\begin{flushleft}
proof:\\
by closed curve integral thm
\begin{align*}
    \int_{C}\frac{f(z)-f(a)}{z-a} dz = 0 \\
\end{align*}
thus
\begin{align*}
    \int_{C} \frac{f(z)}{z-a} dz - \int_{C} \frac{f(a)}{z-a} dz = 0\\
\end{align*}
let $z-a = re^{i\theta}$, $dz = ire^{i\theta} d\theta$, $C_{p}$是圓心$a$的圓 半徑為$p$的圓\\
where $0 < p < r$\\
then
\begin{align*}
    \int_{C_{p}} \frac{d_z}{z-a} = \int_{0}^{2\pi} \frac{ire^{i\theta}}{re^{i\theta}}  d\theta = 2\pi i \\
\end{align*}
所以我們帶入上面的結果\\
\begin{align*}
    \int_{C} \frac{f(a)}{z-a} dz &= f(a) \int_{0}^{2\pi} \frac{1}{re^{i\theta}} ire^{i\theta} d\theta \\
    &= f(a) \int_{0}^{2\pi} i d\theta \\
    &= 2\pi if(a) \\
\end{align*}
\end{flushleft}


\section{liouville's theorem}
如果$f(z)$在整個複數平面上解析且有界,則$f(z)$為常數函數。\\
*entire + bounded $\Rightarrow$ 常數函數\\
proof:\\
假設$|f(z)| \leq M$,則對任意$z_0 \in \mathbb{C}$, $f(z)$在$z_0$的power series展開為
\begin{align*}
    f(z) = \sum_{n=0}^{\infty} c_{n}(z-z_0)^{n} \\  
\end{align*}
其中
\begin{align*}
    c_{n} = \frac{f^{(n)}(z_0)}{n!} = \frac{1}{2\pi i} \int_{|z-z_0|=r} \frac{f(z)}{(z-z_0)^{n+1}} dz \\
\end{align*}
由M-L公式可知
\begin{align*}
    |c_{n}| \leq \frac{M}{r^{n}} \\
\end{align*}
因為$r$可任意大,所以$c_{n} = 0$對所有$n \geq 1$,所以$f(z) = c_0$為常數函數。\\

% \section{Cauchy-Goursat theorem}
% if $f(z)$ is analytic on and inside a simple closed curve $\gamma$, then
% \begin{align*}
%     \int_{\gamma} f(z) dz = 0 \\
% \end{align*}



\end{document}