\documentclass{article}
\usepackage{amsmath}
\usepackage{xeCJK}
\usepackage{amssymb}
\usepackage{indentfirst}
\usepackage{geometry}
\setCJKmainfont{標楷體}

\title{Complex Analysis}
\author{107021211鄭竣元}
\date{June 2022}

\begin{document}

\maketitle

\section{Cauchy-Riemann equation}
A function $f(z)=u(x,y)+iv(x,y)$ , $z=x+iy$ is complex differentiable at $z_0=x_0+iy_0$ if and only if $u$ and $v$ are real differentiable at $(x_0,y_0)$.\\
也就是說$u_{x}=v_{y}$且$u_{y}=-v_{x}$ 等價於 $f(z)$ 在 $z_0$ 可微分。
\section{Entire}

$f(x)$ is called entire if it is differentiable in all $\mathbb{C}$ fields.\\
\\
differentiable 是對點($z_0$),analytic 是對區域 (open set),entire 是對整個 $\mathbb{C}$\\
\\
\\
Ex1. show $e^{z}$ is entire.
\\
\begin{align*} 
e^{z} &=  e^{x+iy} \\ 
&=  e^{x}(\cos{y}+i\sin{y})\\
&= e^{x}\cos{y} + ie^{x}\sin{y}\\
\end{align*}
check by Cauchy-Riemann equation $u_{x}=v_{y}$ and $u_{y}=-v_{x}$ is hold on all $\mathbb{R}$ fields.
thus $e^{z}$ is entire.\\
\\
Ex2. show $e^{z}$ is only entire function satisfied
\begin{align*} 
f(z_{1})+f(z_{2}) &=  f(z_{1})f(z_{2}) \\ 
f(x)&=  e^{x}, x\in \mathbb{R} \\
\end{align*}
let $f(z)=g(x,y)+ih(x,y)$\\

\begin{flushleft}
$\because f(z)$ is entire, $f'(0) =e^{0} = 1$\\
$\therefore  g'(0)+ih'(0) = 1$\\
then $g'(0) = 1, h'(0) = 0$\\
\end{flushleft}

thus $g(x,y)=x, h(x,y)=0$\\

反向用Cauchy-Riemann equation,得到:\\

\begin{align*}
g_{x} &= h_{y} \\
g_{y} &= -h_{x} \\
\end{align*}
we get
\begin{align*}
    f''(x) + f(x) &= 0 \\
\end{align*}
所以$f(x) = c_1sin(x) + c_2cos(x)$,(微分兩次等於自己相反也只有sin,cos的產物)\\
帶入$f'(0) = 1$,得到$c_1 = 1, c_2 = i\longrightarrow f(z) = e^{z}$\\

\section{Convergence of complex series}
Thm1\\
$\sum a_{n} z^{n}=A(z)$ is Convergence at radius $R_{1}$\\
$\sum b_{n} z^{n}=B(z)$ is Convergence at radius $R_{2}$\\
$\Rightarrow \sum_{n=0}^{\infty} (\sum_{k=0}^{n} a_{n}\cdot b_{n-k}) z^{n} =A(z) \cdot B(z)$ is Convergence at radius $min(R_{1},R_{2})$\\
proof: 爆破 斜線排在一起 \\

Thm2\\
power series 微分後收斂半徑不變,且無限可微\\
hint: 展開 微分 用原本的$R$作 Convergence test\\

Thm3 (uniqueness)\\
當$f(z)=\sum_{n=0}^{\infty} c_{n} z^{n}$滿足
\begin{enumerate}
    \item 存在非0數列$\{z_{n}\}$,且 $lim_{n \to \infty} z_{n} = 0$\\
    \item 對於所有$z_{n}$, $f(z_{n}) = 0$\\
\end{enumerate}
則$f(z) \equiv 0$ (identical zero)  ($\{c_{n}\}=0$對所有$n$)\\
*推廣:到任何常數 $c$ 也可以 ($g(z)=f(z)+c$)\\
***他告訴了我們 如果兩個analytic function 在一個accumulated point 相等,則兩個function相等\\

Thm3.1: uniqueness of analytic function\\
如果兩個analytic function在一個收斂數列處處相等,則兩個function相等。\\


\section{Accumulated point}
A point $z_0$ is called an accumulated point of a set $E$ if every neighborhood of $z_0$ contains at least one point of $E$ different from $z_0$ itself.\\
(可透過收縮open set 找到一串收斂數列 即$\{d_n\}$ ,$d_{n+1}$ 在半徑為$d(d_{n},z_0)$的openball裡面找)

\section{M-L formula}
if $f(z)$ is continuous on a curve $\gamma$ with length $L$, and if $|f(z)| \leq M$ on $\gamma$, then
\begin{align*}
    |\int_{\gamma} f(z) dz| \leq ML
\end{align*}

\section{analytic function}
幾個analytic function的性質
\begin{enumerate}
    \item $f(z)$ is analytic at $z_0$ if and only if $f(z)$ can be represented by a convergent power series in some neighborhood of $z_0$.
    \item 如果$f(z)$在一個區域$z_0$內解析,則$f(z)$在$z_0$內無限可微。(power series微分後收斂半徑不變,且無限可微)
\end{enumerate}
所以解析函數 == 無限可微函數 = power series (複數空間的微分超強特性)

\section{Cauchy integral formula}
if $f(z)$ is analytic inside and on a simple closed circle $C=R e^{i\theta}$ 半徑$R$的封閉圓圈, then
\begin{align*}
    f(a) = \frac{1}{2\pi i} \int_{\gamma} \frac{f(z)}{z-a} dz \\    
\end{align*}
\begin{flushleft}
proof:\\
by closed curve integral thm
\begin{align*}
    \int_{C}\frac{f(z)-f(a)}{z-a} dz = 0 \\
\end{align*}
thus
\begin{align*}
    \int_{C} \frac{f(z)}{z-a} dz - \int_{C} \frac{f(a)}{z-a} dz = 0\\
\end{align*}
let $z-a = re^{i\theta}$, $dz = ire^{i\theta} d\theta$, $C_{p}$是圓心$a$的圓 半徑為$p$的圓\\
where $0 < p < r$\\
then
\begin{align*}
    \int_{C_{p}} \frac{d_z}{z-a} = \int_{0}^{2\pi} \frac{ire^{i\theta}}{re^{i\theta}}  d\theta = 2\pi i \\
\end{align*}
所以我們帶入上面的結果\\
\begin{align*}
    \int_{C} \frac{f(a)}{z-a} dz &= f(a) \int_{0}^{2\pi} \frac{1}{re^{i\theta}} ire^{i\theta} d\theta \\
    &= f(a) \int_{0}^{2\pi} i d\theta \\
    &= 2\pi if(a) \\
\end{align*}
\end{flushleft}


\section{Liouville's theorem}
如果$f(z)$entire且有界,則$f(z)$為常數函數。\\
*entire + bounded $\Rightarrow$ 常數函數\\
proof:\\
假設$|f(z)| \leq M$,我們要證明對於任意$a,b \in \mathbb{C}$, $f(a) = f(b)$\\
其中引用 Cauchy integral formula 可得\\
\begin{align*}
    f(a)-f(b)&= \frac{1}{2\pi i} \int_{C} \frac{f(z)}{(z-a)} dz - \frac{1}{2\pi i} \int_{C} \frac{f(z)}{(z-b)} dz\\
    &= \frac{1}{2\pi i} \int_{C}\frac{f(z)(b-a)}{(z-a)(z-b)} dz \\
\end{align*}
又因為$f(z)<M$ , $R=|z|>|a|,|b|$ R我們可以挑的任意大\\
\begin{align*}
    \frac{1}{2\pi i} \int_{C}\frac{f(z)(b-a)}{(z-a)(z-b)} dz < \frac{1}{2\pi i} \frac{M(b-a)}{R^{2}} \cdot 2\pi R = \frac{ M(b-a)}{R} \\
\end{align*}
所以當R挑的足夠大,M也是個常數下\\
\begin{align*}
    f(a)-f(b) < \frac{M(b-a)}{R} = 0 \\
\end{align*}
因此 $f(z)$ 為常數函數\\

Ex1.\\
show that is no nonconstant entire function can satisfied 
\begin{itemize}
    \item $f(z+1) = f(z)$
    \item $f(z+i) = f(z)$
\end{itemize}
for all $z \in \mathbb{C}$\\
proof:\\
用 liouville's theorem\\
Assume $f(z)$ is a  nonconstant entire function\\
then $\forall z \in \mathbb{C}$, $\exists \zeta = x - \left\lfloor x \right\rfloor + i\left( y - \left\lfloor y \right\rfloor \right)$, where $z = x + iy$\\

we pick a set $S = \{x+iy | 0 \leq |x| \leq 1, 0 \leq |y| \leq 1\}$ , ($\zeta \in S$)\\
as we konwn $f(z)$ is continuous on $S$ and $S$ is closed and bounded on $\mathbb{R}^2$, by heine-borel $f(z)$ is compact \\
(沒想到吧,是我 高微 closed + bdd == compact)\\
since $f(S)$ is bounded , $f(z)$ is entire\\
by liouville's theorem, $f(z)$ is constant function ($\rightarrow \leftarrow $)\\

Thus there is no nonconstant entire function can satisfied the above two conditions.\\

\section{Maximum modulus theorem}
if $f(z)$ is analytic and nonconstant in a region $D$, 且無local maximum內點\\

他還有一個孿生姊妹 thm : minimum modulus theorem\\
證明方法就 前面的 cauchy integral formula 所帶出的的 $f(z)$ 會是那個小環的平均值(Cauchy Mean-value thm)。
你不是平面 那你就會一邊高一邊低的感覺\\
也就是說 analytic nonconstantfunction 在bdd區域內不會有local max or min == 極值都在邊邊上\\

\section{Automorphism of unit disk}
    後面常用到的 automorphism of unit disk\\
    $$B_a(z) = \frac{z-a}{1-\overline{a}z}, |a|<1$$\\

\section{Schwarz lemma}
if $f(z)$ is analytic in the unit disk ($|z|<1$) and satisfies $f(0)=0$ and $|f|<1$, then\\
\begin{enumerate}
    \item $|f(z)| \leq |z|$ for all $z$ in the unit disk.
    \item $|f'(0)| \leq 1$
    \item 只要(1.) or (2.) 的等號成立 then iff $f(z) = e^{i\theta} z$.(即換角function)\\
\end{enumerate}

proof:\\
用 Maximum modulus theorem\\
let 

$$g(z) =
\begin{cases}
     \frac{f(z)}{z}, 0<|z|<1\\
     f'(0) = g(0)\\
\end{cases}
$$
then $g(z)$ is analytic(課本6.7) in the unit disk.\\
且 $|g(z)| \leq \frac{1}{r}$ 把$r$限制在unit disk 內 即 $r = 1$\\
by Maximum modulus theorem, $|g(z)|$ has no local maximum in the unit disk
thus $|g(z)| \leq 1$ for all $z$ in the unit disk.\\
hence $|f(z)| \leq |z|$ for all $z$ in the unit disk.\\
白話: 我們不是造出 g(z) 嗎? 我們來想一下他在邊邊的最大值 也就是($r=1$ $f(z) \leq 1$)\\
是不是就小於1 ,又因為他沒有local maximum,所以他在邊邊的值的最大值 一定大於其他內點\\
也就是整個region都小於1\\
講一下等號 , $f'(0) = 1$ or $f(z) = |z|$等價於 $g(z)$ 是 constant\\

Ex1.\\
找出對unit disk fixed point $\alpha$來說,$max_f|f'(\alpha)|$\\
最大的alanlytic function f\\

proof:\\
let 

$$h(z) =
\begin{cases}
     \frac{f(z)-f(\alpha)}{z-\alpha} \cdot \frac{1-z\overline{\alpha}}{1-f(z)\overline{f(\alpha)}} , z \neq \alpha\\
     f'(\alpha) \cdot \frac{1-|\alpha|^2}{1-f(\alpha)}|^2 , z=\alpha\\
\end{cases}
$$

then $h(z)$ is analytic in the unit disk and $|h(z)| \leq 1$\\
by Schwarz lemma, $|h(\alpha)| \leq 1$\\
因此 $|f'(\alpha)| \leq \frac{1-f(\alpha)}{1-|\alpha|^2}$\\

\section{homotopy theorem}
if $f(z)$ is analytic in a region $D$ and if two closed curves $\gamma_1$ and $\gamma_2$ in $D$ are homotopic in $D$, then
\begin{align*}
    \int_{\gamma_1} f(z) dz = \int_{\gamma_2} f(z) dz 
\end{align*}
好像要畫圖才比較好理解,兩條homotopy 就是想像可以用連續函數慢慢拉
直到第一條curve的形狀變成第二條curve (請想像一條固定端點的繩子被拉成兩種不同樣貌)\\
因為這兩個東西可以透過簡單的變數變換,所以積分值相等(用分析幾何的想法)\\

Ex1.\\
證明 在punch plane 中 unit cirle cannot homotopic to a constant curve\\

proof:\\
首先觀察上就不可能了 , punch plane 就是 $\mathbb{C} - \{0\}$ 把 0點挖掉的複數空間\\
我們可以考慮函數 $f(z) = \frac{1}{z}$\\
因為 $f(z)$ 在 punch plane 內解析\\
所以我們可以用 homotopy thm 驗證\\
\begin{align*}
    \frac{1}{2\pi i} = \int_{C_{unit}} \frac{1}{z} dz \neq \int_{\alpha_0} \frac{1}{z} dz = 0\\
\end{align*}
前面是我們的老朋友了\\
後面是constant curve (constant curve 也就是單個點)積分當然是0\\
橡皮筋中釘了一個釘子 $\rightarrow$ 橡皮筋無法縮成一個點\\        

\section{Cauchy closed curve theorem}
if $f(z)$ is analytic in a region $D$ and if $\gamma$ is a smooth closed curve in $D$, then
\begin{align*}
    \int_{\gamma} f(z) dz = 0
\end{align*}
讓複變 變輕鬆的定理之一 閉環積一圈等於0 \\
(內部必須無奇點 奇點的話就要用residue thm那等等再講)\\
那你就可以計算一些以前積分積不出來的東西 透過構造多個閉環疊加\\
堆出你想要的積分區域(讓數學變畫圖遊戲)\\

proof:\\
我們前面提到 analytic function 可等價於 power series\\
所以簡單的一次積分是可以積上去的\\
那積一圈就會變成簡單的 $F(\gamma(0))-F(\gamma(1))$ \\
起終同點 即 $\gamma(0)=\gamma(1)$ 所以\\
$$\int_{C} f(z) dz = 0$$

\section{Laurent expansion}
if $f(z)$ is analytic in the annulus $r<R<|z-z_0|<R_2$, then $f(z)$ can be represented by a convergent power series of the form
\begin{align*}
    f(z) = \sum_{n=0}^{\infty} a_{n}(z-z_0)^{n} + \sum_{n=1}^{\infty} b_{n}(z-z_0)^{-n}
\end{align*}

這樣講可能不好理解,他其實就是強行進行Taylor展開\\
由於整個函數 有部分無法對某特定點微分\\
即 對於a點展開時 由於函數內部擁有 $\frac{1}{(z-a)^n}$ 這樣因式 所以無法展開\\ 
所以我們就把函數分成兩個部分\\
\begin{align*}
    \text{(原函式)} 
    &= \text{(principal part(不可解析))} \times \text{(analytic part(可解析))} \\
    &= \text{(principal part(不可解析))} \times \text{(analytic part (Taylor 展開式))} \\
    &= \text{(Laurent 展開式)}\\
\end{align*}

\section{Cauchy residue theorem}
Suppose \( f \) is analytic in a simply connected domain \( D \) except for isolated singularities at 


\[
z_1, z_2, \dots, z_m.
\]


Let \( \gamma \) be a closed curve not intersecting any of the singularities. Then


\[
\int_{\gamma} f(z)\,dz = 2\pi i \sum_{k=1}^{m} n(\gamma, z_k) \operatorname{Res}(f; z_k),
\]


where \( n(\gamma, z_k) \) denotes the winding number of the curve \( \gamma \) around the point \( z_k \), and \( \operatorname{Res}(f; z_k) \) is the residue of \( f \) at \( z_k \).

簡單來說 對於閉環中有奇點的積分 = $2\pi i$乘上所有奇點的residue和\\
(這個定理可以說是複變中最強大的定理了,因為他可以把你積不出來的東西變成一個個residue相加)\\

proof:\\
對於每個奇點 我們生成一個小閉環 $C_k$ 包住奇點 $z_k$\\
由於閉環的特性 我們可以把原本的閉環積分拆成多個閉環積分相加(closed curve integration theorem)\\
(請好好畫出那些環 再用拓譜的形式 一個一個拉到相互抵消掉 你就會看懂了)\\
如果用畫面來想的話 就是在閉環內數個點 開始長泡泡 泡泡貼泡泡會併成一個大泡泡\\
由於每個泡泡都是順時針方向(你開心也是可以全部改逆時針) 所以接觸的共線段方向相反 會抵消\\
只留下大泡泡的部分\\

至於為什麼是 $2\pi i$ 呢?\\
回到我們的老朋友 Cauchy integral formula\\
\begin{align*}
    \int_{C_{p}} \frac{d_z}{z-a} = \int_{0}^{2\pi} \frac{ire^{i\theta}}{re^{i\theta}}  d\theta = 2\pi i \\
\end{align*}
對於繞著不可解析點的閉環積分 積出來就是 $2\pi i$ (重根記得算兩個)\\

\section{log function}
The complex logarithm is a multi-valued function defined as follows:
\[
\log(z) = \ln|z| + i\arg(z) + 2\pi i k, \quad k \in \mathbb{Z}
\]
log function 在複數空間中是多值的,因為arg function本來就是多值的\\
我們回頭去想 log 的反函數 $e^{z}$\\
$e^{i\theta}$ 函數可以把整個實數域 壓到$[0, 2\pi)$\\
那反過來 我們會把 $[0, 2\pi)$ 送回整個實數域\\
他會有點像螺旋洋芋片那樣 一圈一圈的往上升\\
所以我們會需要一個branch cut 來把他切開\\

所以我們定義了兩個東西 \\
\begin{itemize}
    \item 0 旁邊絕對不要靠近,是一對無限多
    \item branch cut 每次都從 $-\pi$ 到 $\pi$
\end{itemize}
總而言之,在使用log function時,請務必注意所選擇的branch cut(Codomain).\\

\end{document}